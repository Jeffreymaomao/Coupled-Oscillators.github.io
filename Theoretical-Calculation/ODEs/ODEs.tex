\documentclass[12pt]{article}%{revtex4}
%%%%%%%%%%%%
\setlength{\textwidth}{6.9in} \setlength{\textheight}{9.2in}
\hoffset -2.cm \voffset -2.0cm \pagestyle{empty}
\usepackage{bbold}
\usepackage{amsmath}
\usepackage{amssymb}
\usepackage{graphicx} %插入图片的宏包
\usepackage{float} %设置图片浮动位置的宏包
\usepackage{subfigure} %插入多图时用子图显示的宏包
\usepackage{physics} 

%\usepackage{calrsfs}
%\DeclareMathAlphabet{\pazocal}{OMS}{zplm}{m}{n}

\usepackage{tcolorbox}

%\usepackage{fontspec}
%\usepackage{xeCJK} %引用中文字的指令集
%\setCJKfamilyfont{kai}{標楷體}

\usepackage{import}

\usepackage{tikz}

\begin{document}
%\tighten
\begin{center}
{\Large Physics, Classical Dynamics

Couple Oscillator  \#\today 

---------------------------------------------------------------------------------}
\end{center}
	\section{System of Odinary Differential Equation}
	%------------------------------------------------
	\subsection{General Solution}
	For the System
	 \begin{equation}
	 \begin{pmatrix}\ddot{x}_1\\\ddot{x}_2\end{pmatrix}
	 =
	 \begin{pmatrix}
	 \displaystyle - \frac{k_1+k_2}{m_1}	&\displaystyle  \frac{k_2}{m_1}\\[2ex]
	 \displaystyle  \frac{k_2}{m_2} 		&\displaystyle  -\frac{k_2+k_3}{m_2}
	 \end{pmatrix}
	 \begin{pmatrix}x_1\\x_2\end{pmatrix},
	 \end{equation}
	 in the form $\ddot{\vec{x}} = \mathcal{F} \vec{x}$, we solve the eigen problem for matrix $\mathcal{F}$
	  \begin{equation}
	 \left(\mathcal{F}-\lambda\mathcal{I}\right)\mu = 0,
	 \end{equation}
	 we obtain the eigenvalues
	  \begin{equation}
	 \lambda_{1,2} = \frac{\tr(\mathcal{F}) \pm \sqrt{\tr(\mathcal{F})^2-4\det({\mathcal{F}})}}{2},
	 \end{equation}
	 where the trace and determinant of matrix are
	 \begin{equation}
	 \begin{aligned}
	 \tr(\mathcal{F}) &=-\frac{m_2k_1+\left(m_1+m_2\right)k_2+m_1k_3}{m_1m_2},\\
	  \det(\mathcal{F}) &=\frac{k_1k_2+k_2k_3+k_3k_1}{m_1m_2}.
	 \end{aligned}
	 \end{equation}
	 Also, the corresponding eigenvectors are
	 \begin{equation}
	\vec{\mu}_{i} = \begin{pmatrix}
	  	\displaystyle  m_2 \lambda_{i}+ \left(k_2+k_3\right)\\
		\displaystyle  k_2
		\end{pmatrix},\quad i =1,2.
	\end{equation}
	Since the general solution for the equation $\ddot{x} = \lambda x$ is $x\left(t\right) = C_1e^{\sqrt{\lambda} t} + C_2e^{-\sqrt{\lambda} t}$, we define
	\begin{equation}
	\begin{aligned}
	\omega_{i} = \pm\sqrt{\lambda_{i}},	\quad i = 1,2\\
	\end{aligned}
	\end{equation}
	the general solution for this system is
	\begin{equation}
	\vec{x}\left(t\right) = \sum_{i=1}^{2}\left(A_{i}e^{\omega_i t}+ B_{i}e^{-\omega_i t}\right)\vec{\mu}_i.
	\end{equation}
	Also, we define the velocity vector
	\begin{equation}
	\vec{v}\left(t\right) = \frac{d\vec{x}}{dt} = \sum_{i=1}^{2}\left(A_{i}\omega_ie^{\omega_i t}-B_{i}\omega_ie^{-\omega_i t}\right)\vec{\mu}_i.
	\end{equation}
	%------------------------------------------------
	\subsection{Initial Value Problem}
	For initial state at $t=0$, we have initial position vector $\vec{x}\left(0\right)$ and initial velocity vector $\vec{v}\left(0\right)$, which are equal to
	\begin{equation}
	\begin{aligned}
	\vec{x}\left(0\right) &= \sum_{i=1}^{2}\left(A_{i}+ B_{i}\right)\vec{\mu}_i,
	\\
	\vec{v}\left(0\right) &= \sum_{i=1}^{2}\left(A_{i}\omega_i-B_{i}\omega_i\right)\vec{\mu}_i.
	\end{aligned}
	\end{equation}
	repectively. If we expand these 2 condition, we may have
	\begin{equation}
	\begin{aligned}
	\vec{x}\left(0\right) &= \left(A_{1}+ B_{1}\right)\vec{\mu}_1+\left(A_{2}+ B_{2}\right)\vec{\mu}_2
	\\
	\vec{v}\left(0\right) &= \left(A_{1}\omega_1-B_{1}\omega_1\right)\vec{\mu}_2+\left(A_{2}\omega_2-B_{2}\omega_2\right)\vec{\mu}_2
	\end{aligned},
	\end{equation}
	which can be ewwriten in matrix form
	\begin{equation}
	\begin{aligned}
	\vec{x}\left(0\right) &= \begin{pmatrix}\vec{\mu}_{1}&\vec{\mu}_{2}\end{pmatrix}\begin{pmatrix}A_{1}+B_{1}\\A_{2}+B_{2}\end{pmatrix}
	\\
	\vec{v}\left(0\right) &= 
	\begin{pmatrix}\vec{\mu}_{1}&\vec{\mu}_{2}\end{pmatrix}
	\begin{pmatrix}A_{1}\omega_1-B_{1}\omega_1\\A_{2}\omega_2-B_{2}\omega_2\end{pmatrix}
	\end{aligned}.
	\end{equation}
	Since we need to obtain the value of $A_{i}$ and $B_{i}$, using the same matrix to reduce equations
	\begin{equation}
	\begin{aligned}
	\vec{x}\left(0\right) &= 
	\begin{pmatrix}\vec{\mu}_{1}&\vec{\mu}_{2}\end{pmatrix}
	\begin{pmatrix}A_{1}&B_{1}\\A_{2}&B_{2}\end{pmatrix}
	\begin{pmatrix}1\\1\end{pmatrix}
	\\
	\vec{v}\left(0\right) &= 
	\begin{pmatrix}\vec{\mu}_{1}&\vec{\mu}_{2}\end{pmatrix}
	\begin{pmatrix}\omega_1&0\\0&\omega_2\end{pmatrix}
	\begin{pmatrix}A_{1}&B_{1}\\A_{2}&B_{2}\end{pmatrix}
	\begin{pmatrix}1\\-1\end{pmatrix}
	\end{aligned}.
	\end{equation}
	Defining following 2 by 2 matrices in order to reduce the equation
	\begin{equation}
	\hat{\mu} = \begin{pmatrix}\vec{\mu}_1&\vec{\mu}_2\end{pmatrix}, \quad 
	\hat{\omega} = \begin{pmatrix}\omega_1&0\\0&\omega_2\end{pmatrix},\quad \text{and} \quad
	\hat{C} = \begin{pmatrix}A_{1}&B_{1}\\A_{2}&B_{2}\end{pmatrix}.
	\end{equation}
	Plug in, we have
	\begin{equation}
	\vec{x}\left(0\right) = 
	\hat{\mu}\hat{C}
	\begin{pmatrix}1\\1\end{pmatrix},\quad
	\vec{v}\left(0\right) = 
	\hat{\mu}\hat{\omega}\hat{C}
	\begin{pmatrix}1\\-1\end{pmatrix}
	\end{equation}
	Applying the correponding inverse matrix for two equation, we obtain
	\begin{equation}
	\hat{C}\begin{pmatrix}1\\1\end{pmatrix} = \hat{\mu}^{-1}\vec{x}\left(0\right) ,\quad
	\hat{C}\begin{pmatrix}1\\-1\end{pmatrix} = \hat{\omega}^{-1}\hat{\mu}^{-1}\vec{v}\left(0\right).
	\end{equation}
	Also we could stack two equation into one single matrix equation
	\begin{equation}
	\hat{C}\begin{pmatrix}1&1\\ 1&-1\end{pmatrix} = 
	\begin{pmatrix}
		\hat{\mu}^{-1}\vec{x}\left(0\right) & 
		\hat{\omega}^{-1}\hat{\mu}^{-1}\vec{v}\left(0\right)
	\end{pmatrix}.
	\end{equation}
	Applying the inverse matrix from right, solving that
	\begin{equation}
	\hat{C} = \frac{1}{2}
	\begin{pmatrix}
		\hat{\mu}^{-1}\vec{x}\left(0\right) & 
		\hat{\omega}^{-1}\hat{\mu}^{-1}\vec{v}\left(0\right)
	\end{pmatrix}\begin{pmatrix}1&1\\ 1&-1\end{pmatrix}.
	\end{equation}
	%------------------------------------------------
	\subsection{Further Calculation}
	From above result, given a initial condition $\vec{x}\left(0\right)$ and $\vec{v}\left(0\right)$, we colud obtain the general solution for this system, which is 
	\begin{equation}
	\vec{x}\left(t\right) = \sum_{i=1}^{2}\left(A_{i}e^{\omega_i t}+ B_{i}e^{-\omega_i t}\right)\vec{\mu}_i,
	\end{equation}
	where the coefficients can be solved by
	\begin{equation}
	\begin{pmatrix}A_{1}&B_{1}\\A_{2}&B_{2}\end{pmatrix} = 
	\frac{1}{2}
	\begin{pmatrix}
		\hat{\mu}^{-1}\vec{x}\left(0\right) & 
		\hat{\omega}^{-1}\hat{\mu}^{-1}\vec{v}\left(0\right)
	\end{pmatrix}\begin{pmatrix}1&1\\ 1&-1\end{pmatrix}.
	\end{equation}
	After a little bit calculation, matrices $\hat{\mu}^{-1}$ and $\hat{\omega}^{-1}$ can be obtained
	\begin{equation}
	\begin{aligned}
	\hat{\omega}^{-1} &= \begin{pmatrix}1/\omega_1&0\\ 0&1/\omega_2\end{pmatrix}
	\\
	\hat{\mu}^{-1} &= \frac{1}{k_2m_2\left(\lambda_1-\lambda_2\right)} 
	\begin{pmatrix}
		k_2&-m_2 \lambda_2- \left(k_2+k_3\right)\\
		-k_2&m_2 \lambda_1+ \left(k_2+k_3\right)
	\end{pmatrix}
	\end{aligned}
	\end{equation}


	


	




	

	

	 

	 
	 
	 
	 
	 
	 
\end{document}

